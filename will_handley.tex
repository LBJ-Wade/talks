\documentclass[aspectratio=169]{beamer}
\usepackage{will_handley}
\usepackage{layouts}
\newcommand{\mean}[2][]{\left\langle#2\right\rangle_{#1}}

% Commands
% --------
% - \arxiv{arxiv number}
% - \cols{width}{lh column}{rh column}
% -  \begin{fig(left|right)}[fractional width (e.g 0.6) ]{name of image}
%        content of other column
%    \end{fig(left|right)}

% Talk details
% ------------
\title{Bayesian methods for quantifying global parameter tensions between cosmological datasets}
%\subtitle{<+subtitle+>}
\date{February 2021}

% Abstract
% --------
% The discrepancy between inferences and measurements of the Hubble constant
% arising from early and late-time datasets is hard to miss. In other contexts
% however, parameter tensions are more challenging to visualise and quantify.
% Topical examples include:
% 
% (a) tensions between cosmic microwave background and weak lensing data, since
% the tension arises in a non-trivial combination of parameters including
% $\sigma_8$ and $\Omega_m$
% 
% (b) tensions arising from or resolved by extensions to $\Lambda$CDM such as
% variable dark energy and/or spatial curvature which have non-linear
% degeneracies.
% 
% In these cases the parameter tensions may be hidden or over-emphasised when
% examining marginalised plots derived from a much higher-dimensional parameter
% space (such as those produced by getdist or corner). More advanced 'global'
% metrics are required to determine whether or not two experiments are in
% tension, or whether a proposed solution actually resolves the discrepancy.
% 
% This talk will highlight the recent statistical and computational advances in
% the theory of quantifying parameter tensions using a Bayesian and
% information-theoretic approach, with a focus on the Suspiciousness statistic.
% 
% These techniques will be demonstrated with applications to modern datasets
% such as Planck, ACT, SPT, DES, KiDS and SH0ES, and the current computational
% state-of-the-art toolkits will be indicated. The headline result is a
% technique and interpretation for computing the Suspiciousness statistic from
% MCMC chains alone, without the requirement of an explicit Bayesian evidence
% computation.
% 
% https://arxiv.org/abs/2007.08496
% https://arxiv.org/abs/1908.09139
% https://arxiv.org/abs/1903.06682
% https://arxiv.org/abs/1902.04029



\begin{document}

\begin{frame}
    \titlepage
\end{frame}

\begin{frame}
    \frametitle{Cosmological parameter tensions}
    \begin{figright}[0.51]{figures/H0}
        \begin{itemize}
            \item Measurements $H_0$ differ between early and late time observations \arxiv{1907.10625} 
            \item ``Tension'' means a disagreement between different datasets on the inferred value of model parameters.
            \item The presence of tension indicates an error in the model or at least one of the datasets.
            \item It is statistically incorrect to combine datasets when they are in tension.
        \end{itemize}
    \end{figright}
\end{frame}

\begin{frame}
    \frametitle{The importance of global measures}
    \begin{columns}
        \begin{column}{0.5\textwidth}
        textwidth in in: \printinunitsof{in}\prntlen{\textwidth}

        textheight in in: \printinunitsof{in}\prntlen{\textheight}
        \arxiv{2007.08496}
        \end{column}
        \begin{column}{0.5\textwidth}
            \includegraphics<1>{figures/DES_planck_1}
            \includegraphics<2>{figures/DES_planck_2}
        \end{column}
    \end{columns}
\end{frame}

\begin{frame}
    \frametitle{The perils of marginalisation}
    \begin{columns}
        \begin{column}{0.5\textwidth}
            \begin{align*}
                t=&-\Omega_b h^2 + 0.022 \Omega_c h^2 + 34\theta_{MC} -0.092 \tau\\ &+ 0.05 {\rm{ln}}(10^{10} A_s) + 0.067 n_s
            \end{align*}
        \end{column}
        \begin{column}{0.5\textwidth}
            \includegraphics<1>[width=\textwidth]{figures/act_planck}
            \includegraphics<2>{figures/act_planck_t}
        \end{column}
    \end{columns}
\end{frame}

\begin{frame}
    \frametitle{Bayesian language}
    \begin{itemize}
        \item Datasets $A$ and $B$ (e.g. \textit{Planck} and S$H_0$ES, or \textit{Planck} and DES)
        \item Model $M$ (e.g. $\Lambda$CDM)
        \item Parameter(s) $\theta$ (e.g. $H_0$ or $(\Omega_m,S_8)$) 
        \item Likelihoods $P(A|\theta)$, $P(B|\theta)$ 
        \item Given a prior $P(\theta)$
        \item Evaluate representations (samples) from posteriors $P(\theta|B)$, $P(\theta|A)$.

    \end{itemize}

\end{frame}

\begin{frame}
    \frametitle{Evidence, KL divergence and model dimensionality}
    \arxiv{1903.06682}

    \[ \log\mathcal{Z} = \langle\log L\rangle_\mathcal{P} - \mathcal{D}  \]

    \[d = 2\times\mathrm{var}(\log L)_\mathcal{P} \] 
\end{frame}

\begin{frame}
    \frametitle{Suspiciousness}
    <+Content+>
    \arxiv{1902.04029}

    Gaussian case reduces to the standard Malhanobis distance
\end{frame}


\begin{frame}
    \frametitle{Curvature tension $\Omega_K$}
    \begin{columns}
        \begin{column}{0.5\textwidth}
        Handley~\arxiv{1908.09139}
        Di Valentino et al~\arxiv{1911.02087}
        Efsthathiou \& Gratton~\arxiv{2002.06892}
        \end{column}
        \begin{column}{0.5\textwidth}
            \includegraphics<1>{figures/curvature_1}
            \includegraphics<2>{figures/curvature_2}
            \includegraphics<3>{figures/curvature_3}
        \end{column}
    \end{columns}
\end{frame}

\begin{frame}
    \frametitle{Summary}
    \begin{enumerate}
        \item 
    \end{enumerate}
\end{frame}

\end{document}
