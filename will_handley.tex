\documentclass[aspectratio=169]{beamer}
\usepackage{will_handley}
\usepackage{layouts}
\newcommand{\mean}[2][]{\left\langle#2\right\rangle_{#1}}
%\usepackage{calc}  


% Commands
% --------
% - \arxiv{arxiv number}
% - \cols{width}{lh column}{rh column}
% -  \begin{fig(left|right)}[fractional width (e.g 0.6) ]{name of image}
%        content of other column
%    \end{fig(left|right)}

% Talk details
% ------------
\title{Bayesian methods for quantifying global parameter tensions between cosmological datasets}
%\subtitle{<+subtitle+>}
\date{24\textsuperscript{th} February 2021}

% Abstract
% --------
% The discrepancy between inferences and measurements of the Hubble constant
% arising from early and late-time datasets is hard to miss. In other contexts
% however, parameter tensions are more challenging to visualise and quantify.
% Topical examples include:
% 
% (a) tensions between cosmic microwave background and weak lensing data, since
% the tension arises in a non-trivial combination of parameters including
% $\sigma_8$ and $\Omega_m$
% 
% (b) tensions arising from or resolved by extensions to $\Lambda$CDM such as
% variable dark energy and/or spatial curvature which have non-linear
% degeneracies.
% 
% In these cases the parameter tensions may be hidden or over-emphasised when
% examining marginalised plots derived from a much higher-dimensional parameter
% space (such as those produced by getdist or corner). More advanced 'global'
% metrics are required to determine whether or not two experiments are in
% tension, or whether a proposed solution actually resolves the discrepancy.
% 
% This talk will highlight the recent statistical and computational advances in
% the theory of quantifying parameter tensions using a Bayesian and
% information-theoretic approach, with a focus on the Suspiciousness statistic.
% 
% These techniques will be demonstrated with applications to modern datasets
% such as Planck, ACT, SPT, DES, KiDS and SH0ES, and the current computational
% state-of-the-art toolkits will be indicated. The headline result is a
% technique and interpretation for computing the Suspiciousness statistic from
% MCMC chains alone, without the requirement of an explicit Bayesian evidence
% computation.
% 
% https://arxiv.org/abs/2007.08496
% https://arxiv.org/abs/1908.09139
% https://arxiv.org/abs/1903.06682
% https://arxiv.org/abs/1902.04029



\begin{document}

\begin{frame}
    \titlepage
\end{frame}

\begin{frame}
    \frametitle{Cosmological parameter tensions}
    \begin{figright}[0.51]{figures/H0}
        \begin{itemize}
            \item Measurements $H_0$ differ between early and late time observations \arxiv{1907.10625} 
            \item ``Tension'' means a disagreement between different datasets on the inferred value of model parameters.
            \item The presence of tension indicates an error in the model and/or at least one of the datasets.
            \item It is statistically incorrect to combine datasets when they are in tension.
        \end{itemize}
    \end{figright}
\end{frame}

\begin{frame}
    \frametitle{The importance of global measures of tension}
    \begin{columns}
        \begin{column}{0.5\textwidth}
            \begin{itemize}
                \item In other situations the discrepancy doesn't exist in a single interpretable parameter
                \item For example: DES+\textit{Planck} \arxiv{1902.04029} 
                \item Are these two datasets in tension?
                \item Can we confidently combine them?
                \item There are a lot more parameters -- are we sure that we've chosen wisely?
            \end{itemize}
        \end{column}
        \begin{column}{0.5\textwidth}
            \includegraphics<1>{figures/DES_planck_1}
            \includegraphics<2>{figures/DES_planck_2}
        \end{column}
    \end{columns}
\end{frame}

\begin{frame}
    \frametitle{The perils of manual marginal inspection}
    \begin{columns}
        \begin{column}{0.5\textwidth}
            \begin{itemize}
                \item If you have enough parameters, then you might expect that tensions would naturally arise in some combinations by chance.
                \item For example, if you take ACT and \textit{Planck}, and construct a linear combination of parameters in maximum tension:
                    \begin{align*}
                        t=&-\Omega_b h^2 + 0.022 \Omega_c h^2 + 34\theta_{MC} -0.092 \tau\\ &+ 0.05 {\rm{ln}}(10^{10} A_s) + 0.067 n_s
                    \end{align*}
                \item In general you would expect such a parameter to be in $\sim \sqrt{d}-\sigma$ tension \arxiv{2007.08496}
            \end{itemize}
        \end{column}
        \begin{column}{0.5\textwidth}
            \includegraphics<1>[width=\textwidth]{figures/act_planck}
            \includegraphics<2>{figures/act_planck_t}
        \end{column}
    \end{columns}
\end{frame}

\begin{frame}
    \frametitle{Bayesian language}
    \textbf{Notation}
    \begin{description}[leftmargin=!,labelwidth=200pt]
        \item [Datasets:] $A$ and $B$ (e.g. \textit{Planck} and DES)
        \item [Model:] $M$ (e.g. $\Lambda$CDM)
        \item [Parameters:] $\theta$ (e.g. $(\Omega_m,\sigma_8)$) 
        \item [Likelihoods:] $\mathcal{L}$: $P(A|\theta)$, $P(B|\theta)$ 
    \end{description}
    \textbf{Inference}
    \begin{description}[leftmargin=!,labelwidth=200pt]
        \item [Prior:] $\pi$: $P(\theta)$
        \item [Posteriors:]  $\mathcal{P}$: $P(\theta|B)$, $P(\theta|A)$, (evaluate samples).
        \item [Bayesian evidences:] $\mathcal{Z}=\mean[\pi]{\mathcal{L}}$, $P(A)$, $P(B)$ \hfill\arxiv{1506.00171}
        \item [Bayes theorem:] $\mathcal{L}\times\pi = \mathcal{P}\times\mathcal{Z}$
    \end{description}
    \textbf{Anatomy}
    \begin{description}[leftmargin=!,labelwidth=200pt]
        \item [Kullback--Leibler divergence:] $\mathcal{D}=\mean[\mathcal{P}]{\log\mathcal{P}/\pi}\qquad\sim \log \mathrm{Vol}(\pi)/\mathrm{Vol}(\mathcal{P})$
            \hfill\arxiv{1902.04029}
        \item [Model dimensionality:] $d = 2\times\mathrm{var}(\log\mathcal{L})$      \hfill \arxiv{1903.06682}
        \item [Occam's razor equation:]
        $ \log\mathcal{Z} = \langle\log \mathcal{L}\rangle_\mathcal{P} - \mathcal{D}  $
        \hfill\arxiv{2102.11511}
        \\\hfill (Released today by Hergt et al)
    \end{description}

\end{frame}
\begin{frame}
    \frametitle{Suspiciousness statistic \arxiv{1902.04029} \arxiv{2007.08496}}

    \begin{itemize}
        \item The natural Bayesian measure of tension is the Bayes ratio
            \begin{equation}
                \mathcal{R} = \frac{\mathcal{Z}_{AB}}{\mathcal{Z}_A\mathcal{Z}_B} = \frac{P(A,B)}{P(A)P(B)} = \frac{P(A|B)}{P(A)}= \frac{P(B|A)}{P(B)}
            \end{equation}
        \item $\mathcal{R}$ is prior dependent, one can artificially reduce tension by drawing arbitrarily wide priors.
        \item Can remove this prior dependency by dividing out the KL-dependent Occam factor to give a ``Suspiciousness'', computable from three MCMC chains:
            \begin{equation}
                \log S = \mean[\mathcal{P}_{AB}]{\log L_{AB}} - \mean[\mathcal{P}_{A}]{\log L_{A}}- \mean[\mathcal{P}_{B}]{\log L_{B}}
            \end{equation}
        \item Can be intepreted as the maximum Bayes ratio $\mathcal{R}$ allowed by reasonable priors.
        \item In the Gaussian case it is related to the usual Malhanobis distance
            \begin{equation}
                \log S = \frac{d}{2} - \frac{1}{2}(\mu_A-\mu_B)^T(\Sigma_A+\Sigma_B)^{-1}(\mu_A-\mu_B) 
            \end{equation}
            which can be used to calibrate it into a tension probability and ``$\sigma$'' quantification.
    \end{itemize}
\end{frame}

\begin{frame}
    \frametitle{Curvature tension $\Omega_K$}
    \begin{columns}
        \begin{column}{0.5\textwidth}
            \begin{itemize}
                \item $\Lambda$CDM assumes the universe is flat
                \item If you allow $\Omega_K\ne0$, \textit{Planck} (\texttt{plikTTTEEE}) has a moderate preference for closed universes (50:1)
                \item \textit{Planck}+CMB lensing +BAO strongly prefer a flat universe
                \item But, \textit{Planck} vs lensing is 2.5$\sigma$ in tension, and Planck vs BAO is 3$\sigma$.
                \item This is reduced if $\texttt{plik}\to\texttt{camspec}$
                    \begin{itemize}
                        \item Di Valentino et al~\arxiv{1911.02087}
                        \item Handley~\arxiv{1908.09139}
                        \item Efsthathiou \& Gratton~\arxiv{2002.06892}
                    \end{itemize}
                \item BAO and lensing summary statistics and compression strategy assume $\Lambda$CDM.
            \end{itemize}
        \end{column}
        \begin{column}{0.5\textwidth}
            \includegraphics<1>{figures/curvature_1}
            \includegraphics<2>{figures/curvature_2}
            \includegraphics<3>{figures/curvature_3}
        \end{column}
    \end{columns}
\end{frame}

\begin{frame}
    \frametitle{Summary}
    \centerline{
        \begin{tabular}{llrl}
            Data & Model & Tension & Reference\\
            \hline
            DES vs Planck& $\Lambda$CDM & 2.1$\sigma$ & \arxiv{1902.04029}\\
            ACT vs Planck+SPT& $\Lambda$CDM & 2.8$\sigma$ & \arxiv{2007.08496}\\
            CMB lensing vs Planck& $K\Lambda$CDM & 2.5$\sigma$ & \arxiv{1908.09139}\\
            BAO vs Planck& $K\Lambda$CDM & 3$\sigma$ & \arxiv{1908.09139}\\
        \end{tabular}
    }

	Slides, figures and plotting code available at:
	\url{https://github.com/williamjameshandley/talks/tree/tehran\_2021}
\end{frame}

\end{document}
