\documentclass[%
    handout
]{beamer}
\usepackage{graphicx} % For including single page pdfs
\usepackage{bm}       % bold math
\usepackage{pgffor}   % for loop
\usepackage{tikz}
\usepackage{multimedia}
\usepackage{layouts}
\usepackage{hyperref}
\usepackage{cambridge_lecture}

\usepackage{calculator}

\newcommand{\cols}[3][0.5]{%
    \SUBTRACT{1.}{#1}{\wdthb}
    \begin{columns}
        \begin{column}{#1\textwidth}
            #2
        \end{column}
        \begin{column}{\wdthb\textwidth}
            #3
        \end{column}
    \end{columns}
}

\newcommand{\figname}{}
\newenvironment{figright}[2][0.5]{%
    \renewcommand{\figname}{#2}
    \SUBTRACT{1.}{#1}{\wdthb}
    \begin{columns}
        \begin{column}{#1\textwidth}
        }{%
        \end{column}
        \begin{column}{\wdthb\textwidth}
            \includegraphics[width=\textwidth]{\figname}
        \end{column}
    \end{columns}
}

\newcommand{\dfigname}{}
\newenvironment{dfigright}[3][0.5]{%
    \renewcommand{\figname}{#2}
    \renewcommand{\dfigname}{#3}
    \SUBTRACT{1.}{#1}{\wdthb}
    \begin{columns}
        \begin{column}{#1\textwidth}
        }{%
        \end{column}
        \begin{column}{\wdthb\textwidth}
            \includegraphics[width=\textwidth]{\figname}
            \includegraphics[width=\textwidth]{\dfigname}
        \end{column}
    \end{columns}
}



\newenvironment{figleft}[2][0.5]{%
    \SUBTRACT{1.}{#1}{\wdthb}
    \begin{columns}
        \begin{column}{#1\textwidth}
            \includegraphics[width=\textwidth]{#2}
        \end{column}
        \begin{column}{\wdthb\textwidth}
        }{%
        \end{column}
    \end{columns}
}

\newenvironment{dfigleft}[3][0.5]{%
    \SUBTRACT{1.}{#1}{\wdthb}
    \begin{columns}
        \begin{column}{#1\textwidth}
            \includegraphics[width=\textwidth]{#2}
            \includegraphics[width=\textwidth]{#3}
        \end{column}
        \begin{column}{\wdthb\textwidth}
        }{%
        \end{column}
    \end{columns}
}


\newcounter{numimages}

\newenvironment{multifig}[1]{%
    \begin{frame}
        \pdfximage{#1}%
        \setcounter{numimages}{\the\pdflastximagepages}
        \addtocounter{numimages}{-1}

        \begin{tikzpicture}[remember picture, overlay]
            \foreach \pagenum in {1,...,\thenumimages} {%
                \node<handout:0|beamer:\pagenum>[anchor=center] at (current page.center) {
                \includegraphics[width=\textwidth,page=\pagenum]{#1}}; 
            }
            \addtocounter{numimages}{1}
            \node<handout:1|beamer:\thenumimages>[anchor=center] at (current page.center) {
            \includegraphics[width=\textwidth,page=\thenumimages]{#1}}; 
        \end{tikzpicture}
    }{%
    \end{frame}
}


\newcommand{\lik}{\mathcal{L}}
\newcommand{\posterior}{\mathcal{P}}
\newcommand{\prior}{\pi}
\newcommand{\ev}{\mathcal{Z}}

\newcommand{\prob}{\mathrm{P}}


\title{Nested Sampling}
\subtitle{An efficient and robust Bayesian inference tool\\ for cosmology and particle physics }
\author[Handley] % (optional, for multiple authors)
{Will Handley\\ \small{wh260@cam.ac.uk}}
\institute[University of Cambridge] % (optional)
{%
Astrophysics Group \\
Cavendish Laboratory \\
University of Cambridge
}
\date{Nov 6, 2018}


\begin{document}

\begin{frame}
    \titlepage{}
\end{frame}

\begin{frame}
    \frametitle{Outline}
    \tableofcontents
\end{frame}

\section{What is nested sampling}
\begin{frame}
    \frametitle{What do I mean by sampling?}
\begin{figright}[0.6]{./figures/MCMCTraversal.png}
    \begin{itemize}
        \item Sampling is the process of generating $D$-dimensional points $\theta=(\theta_1,\ldots,\theta_D)$ drawn from probability distribution $P(\theta)$.
        \item $P(\theta)$ is a-priori unknown, and may be expensive to evaluate.
        \item The name of the game is use as few calls to $P$ as possible.
        \item Points need not be independent, and indeed normally only need $\sim\mathcal{O}(12)$ for most inference purposes.
    \end{itemize}
\end{figright}

\end{frame}

\begin{frame}
    \frametitle{Challenges in sampling techniques}
\begin{figright}[0.4]{./figures/Himmelblau_contour.png}
    \begin{itemize}
        \item Multimodality
        \item Burn-in
        \item Convergence diagnosis
        \item Correlation/Degeneracy 
        \item Parallelisation
        \item Phase-transitions
        \item High dimensions
    \end{itemize}
\end{figright}

\end{frame}

\begin{frame}
  \frametitle{Nested Sampling} 
  \begin{itemize}
      \item Completely new approach to sampling:
  \end{itemize}

  Maintain a set $S$ of $n$ samples, which are sequentially updated:

  \begin{description}
    \item[$S_0$:] Generate $n$ samples uniformly over the space.
    \item[$S_{n+1}$:] Delete the lowest probability sample in $S_{n}$, and replace it with a new sample with higher probability
  \end{description}

  \begin{itemize}
      \item This generates a \emph{run} of discarded points.
      \item $n\sim\mathcal{O}(10s-1000s)$
      \item Requires one to be able to uniformly within a region, subject to a {\em hard probability constraint}.
      \item John Skilling's original paper: \href{https://projecteuclid.org/euclid.ba/1340370944}{euclid.ba/1340370944}
  \end{itemize}
\end{frame}

\begin{multifig}{figures/nested_sampling.pdf}
\end{multifig}

\begin{frame}
  \frametitle{How is Nested Sampling used?} 
  \begin{itemize}
      \item Nested sampling generates a \emph{run} of discarded points
      \item These points can be weighted in post-processing to give:
          \begin{itemize}
              \item Posterior samples
              \item Bayesian Evidence (marginal likelihoods)
              \item Kullback Liebler divergence
              \item Partition function
          \end{itemize}
      \item This is possible because the nested sampling scheme is a probabilistic integrator, allowing one to estimate the \emph{density of states}.
  \end{itemize}
\end{frame}


\begin{multifig}{figures/lesbesgue.pdf}
    \frametitle{Probabilistic integration}
\end{multifig}

\begin{frame}
    \frametitle{Estimating the density of states} 

    \begin{itemize}
        \item If number of live points $n=100$, each uniformly sampled point sits in a shell $\approx 1\%$ of volume of outer most contour
        \item At each iteration, contour shrinks in volume by  $\approx 1/n$.
            \begin{equation}
                \ev \approx \sum_i \Delta\lik_i X_{i},\qquad
                X_{i+1} \approx \frac{n}{n+1}X_i, \qquad X_{0} = 1
                \nonumber
            \end{equation}
        \item Nested sampling zooms in to the peak of the posterior {\em exponentially}.
        \item In fact, we perform precise inference on the volumes:
            \begin{equation}
                P(X_{i+1}|X_{i}) = n{[X_{i+1}/X_{i}]}^{n-1}
                \nonumber
            \end{equation}
        \item Posterior weights are $\mathcal{P}_i = \mathcal{L}_i\times (X_i-X_{i-1})$
    \end{itemize}
\end{frame}

\begin{frame}
    \frametitle{Key advantages of nested sampling} 
    \begin{itemize}
        \item The density of states (prior volume estimation) is the missing piece in inference, normally avoided/cancelled in traditional methods.
        \item Allows numerical computation of Bayesian Evidence \& KL divergence.
        \item At each iteration, the set of live points enables self-tuning:
            \begin{itemize}
                \item Clustering (handling multi-modality)
                \item Correlation estimation (constructing proposal distributions)
            \end{itemize}
        \item The sampling process is athermal, and invariant under monotonic tranformations of the sampled distribution $L(\theta)\to f(L(\theta))$:
            \begin{equation}
                E(\theta) = -\log L(\theta) \qquad P(\theta) = \frac{1}{Z(\beta)}e^{-\beta E(\theta)}
            \end{equation}
        \item By appropriate re-weighting, we can post-process the posterior samples to be at any temperature.
    \end{itemize}
\end{frame}

\begin{frame}
    \frametitle{How do we perform nested sampling?} 
    \begin{itemize}
        \item Requires one to be able to uniformly within a region, subject to a {\em hard probability constraint}.
        \item Two main codes that implement traditional NS\@:
            \begin{description}
                \item[MultiNest] \href{https://arxiv.org/abs/0809.3437}{arXiv:0809.3437} 
                \item[PolyChord] \href{https://arxiv.org/abs/1506.00171}{arXiv:1506.00171} 
            \end{description}
        \item Alternative frameworks:
            \begin{description}
                \item[Diffusive NS] \href{https://arxiv.org/abs/0912.2380}{arXiv:0912.2380} 
                \item[SE NS]\href{https://arxiv.org/abs/1402.6306}{arXiv:1402.6306} 
                \item[Dynamic NS] \href{https://arxiv.org/abs/1704.03459}{arXiv:1704.03459} 
            \end{description}
    \end{itemize}

\end{frame}


\section{MultiNest}
\begin{multifig}{figures/multinest.pdf}
    \frametitle{MultiNest}
\end{multifig}
\begin{frame}
    \frametitle{MultiNest} 
    \begin{itemize}
        \item \href{https://arxiv.org/abs/0809.3437}{arXiv:0809.3437} 
        \item Uses a set of overlapping ellipsoids to approximate the shape of the contours
        \item Rejection samples from this representation.
        \item Maximally efficient in low dimensions\ldots
        \item \ldots exponentially bad in high dimensions.
        \item Transition is distribution-dependent (as low as $5$ or as high as $60$).
        \item John Skilling originally anticipated/advocated using MCMC-style chain-based approach.
    \end{itemize}
\end{frame}

\section{PolyChord}
\begin{multifig}{figures/polychord.pdf}
    \frametitle{PolyChord}
\end{multifig}

\begin{frame}
    \frametitle{PolyChord} 
    \begin{itemize}
        \item \href{https://arxiv.org/abs/1506.00171}{arXiv:1506.00171}
        \item Uses slice sampling to generate new points: \href{https://projecteuclid.org/euclid.aos/1056562461}{euclid.aos/1056562461} 
        \item Each step requires $\sim\mathcal{O}(3-5)$ evaluations, need $N\sim\mathcal{O}(D)$ steps per chain to decorrelate from start point.
        \item Worse than MultiNest in low dimensions, exponentially more efficient in high dimensions.
        \item Chain-based $\Rightarrow$ allows exploitation of fast-slow hierarchy.
        \item Under active maintenance/development
    \end{itemize}
\end{frame}

\section{Advances in Nested Sampling}
\begin{frame}
    \frametitle{Further advances in nested sampling}
    \begin{itemize}
        \item Dynamic nested sampling
            \begin{itemize}
                \item \href{https://arxiv.org/abs/1704.03459}{arXiv:1704.03459} 
                \item Allows one to refine a run and generate more points, e.g.\ in the posterior bulk, or prior tails.
            \end{itemize}
        \item Consistency checking
            \begin{itemize}
                \item \href{https://arxiv.org/abs/1804.06406}{arXiv:1804.06406} 
                \item Unweaving runs allows for cross-checking and testing for imperfect contour sampling.
            \end{itemize}
        \item Diffusive nested sampling
            \begin{itemize}
                \item \href{https://arxiv.org/abs/0912.2380}{arXiv:0912.2380} 
                \item Fuzzy contours represent an alternative approach to nested sampling
            \end{itemize}
        \item PolyChord 2.0
            \begin{itemize}
                \item \href{https://arxiv.org/abs/0912.2380}{arXiv:0912.2380} 
                \item Under active develoment, promises $\sim\mathcal{O}(D)$ speed-up
            \end{itemize}
    \end{itemize}
\end{frame}

\begin{frame}
    \frametitle{Takeaway points}
    \begin{itemize}
        \item Nested sampling is far more than a posterior sampler.
        \item To do nested sampling in high dimensions, you cannot use MultiNest.
        \item Fast-slow hierarchies have proven extremely useful in speeding up Planck, AMI and DES analyses.
        \item PolyChord is available on GitHub: \href{https://github.com/PolyChord/PolyChordLite}{github.com/PolyChord/PolyChordLite}
        \item PolyChord interfaces are available for CosmoMC, cosmosis, MontePython and GAMBIT\@.
    \end{itemize}
\end{frame}

\end{document}
